\documentclass[12pt,a4paper]{report}
\usepackage[utf8]{inputenc}
\usepackage[english]{babel}
\usepackage[T1]{fontenc}
\usepackage{amsmath}
\usepackage{amsfonts}
\usepackage{amssymb}
\usepackage{multicol}
%\usepackage{tabulary}
\usepackage{listings}
\usepackage{color}
\usepackage{graphicx}
%\usepackage{tcolorbox} % color boxes for definition
\usepackage{nameref} % reference to chapters with \nameref{}, need a \label{}

% Makes references hyperlinks in PDF output
\usepackage[hidelinks]{hyperref}
\usepackage{parskip}
\usepackage[left=2cm,right=2cm,top=2cm,bottom=2cm]{geometry}

\renewcommand{\thesection}{\Roman{section}}
\renewcommand{\thesubsection}{\arabic{subsection}}

\makeatletter
\def\maketitle{
%null
\begin{minipage}{0.54\textwidth}
\begin{flushleft} \large
\LARGE \@author
\end{flushleft}
\end{minipage}
\begin{minipage}{0.4\textwidth}
\begin{flushright} \large
%\includegraphics[width=6cm]{telecomnancy.jpg} %ou image.png, .jpeg etc.
\end{flushright}
\end{minipage}
  \vfill
  \begin{center}\leavevmode
    \normalfont
    {\LARGE \@title\par}%
    {\Large \@date\par}
    \vskip 1cm
    %
  \end{center}%
  \vfill
  \hfill
  
\begin{minipage}{0.54\textwidth}
\begin{flushleft} \large
% texte
\includegraphics[width=6cm]{telecomnancy.jpg}


\end{flushleft}
\end{minipage}
\begin{minipage}{0.4\textwidth}
\begin{flushright} \large
\includegraphics[width=6cm]{univ-lorraine.jpg}
\end{flushright}
\end{minipage}
  \cleardoublepage
  }
  
\makeatother

\title{English Project \\ \textbf{TOEIC Trainer}}
\author{\textsc{Clément Schanen} \\ \textsc{Thibaut Smith}}
\date{2013-2014}

    %\renewcommand*\thesection{\arabic{section}}
    
    
\makeatletter
\g@addto@macro\@floatboxreset\centering
\makeatother


\begin{document}
\pagestyle{empty} % removes page numbering

\maketitle % implicit pagebreak

\pagestyle{plain}
\setcounter{page}{2}

\tableofcontents
\section*{Introduction}
As part of TELECOM Nancy's English course, we, \textsc{Clément Schanen} and
\textsc{Thibaut Smith} tried our best to come up with an interesting and
useful product for the students of TELECOM Nancy.

One of our objective for this project was to make a sustainable product that
would allow for the community using it to refine it themselves and adding
materials to study.

\pagebreak
\section{Website's Handbook}
% 2-5 pages
% Présentation de l'outil, mode d'emploi

The product that we came up with is a website\cite{TOEICTrainer}. In order to complete the pre-requisits that we wanted, we thought a website could reach everyone at school, and beyond. You can find the sources on GitHub\cite{github_tc}.

\subsection{What you can do with this website}

The main goal of this website is helping people improved their english level by doing TOEIC like exercises. Our exercises are divided in two parts : the listening part and the reading part. And every part is also divided in two exercises, so a user can do four different exercises : Photographs, Question/Answer, Incomplete sentences and Incomplete documents.

Moreover, this website is collaborative, so every people can submit new contains (text, audio or picture) to build new exercises

\subsection{Listening exercices}
\subsection{Reading exercices}
The TOEIC has three kinds of exercices in the reading session. We implemented
two of them:

\begin{itemize}
\item Fill the blanks in an incomplete sentences
\item \textbf{Fill the blanks in a document}
\item \textbf{Use documents to answer questions about them}
\end{itemize}

The bold lines are the exercices you can find on the website.

\pagebreak

\section{Documentation technique}
% 2-5 pages
\subsection{Mise en oeuvre}

\subsection{Développement}
\subsection{Problèmes rencontrés}


\section*{Conlusion}
This project made us learn a great deal of things in web development, and we
can be proud to say that we have now a better insight of the most famous tools
and programming languages currently used in business.
 
We tried analysing what were the requirements of such a project to come up with
an interesting result that would not only help people studying for the TOEIC
examination, but also a product that would not expire as fast as a project gone
still. This is how we thought of giving everyone the possibility to submit new
content and giving the ones that are more at ease with the english language.


\bibliographystyle{plain} % biblio style
\bibliography{biblio} % to load biblio.bib


\end{document}