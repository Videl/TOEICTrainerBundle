\documentclass[12pt,a4paper]{report}
\usepackage[utf8]{inputenc}
\usepackage[english]{babel}
\usepackage[T1]{fontenc}
\usepackage{amsmath}
\usepackage{amsfonts}
\usepackage{amssymb}
\usepackage{multicol}
%\usepackage{tabulary}
\usepackage{listings}
\usepackage{color}
\usepackage{graphicx}
%\usepackage{tcolorbox} % color boxes for definition
\usepackage{nameref} % reference to chapters with \nameref{}, need a \label{}

% Makes references hyperlinks in PDF output
\usepackage[hidelinks]{hyperref}
\usepackage[left=2cm,right=2cm,top=2cm,bottom=2cm]{geometry}

\renewcommand{\thesection}{\Roman{section}}
\renewcommand{\thesubsection}{\arabic{subsection}}

\makeatletter
\def\maketitle{
%null
\begin{minipage}{0.54\textwidth}
\begin{flushleft} \large
\LARGE \@author
\end{flushleft}
\end{minipage}
\begin{minipage}{0.4\textwidth}
\begin{flushright} \large
%\includegraphics[width=6cm]{telecomnancy.jpg} %ou image.png, .jpeg etc.
\end{flushright}
\end{minipage}
  \vfill
  \begin{center}\leavevmode
    \normalfont
    {\LARGE \@title\par}%
    {\Large \@date\par}
    \vskip 1cm
    %
  \end{center}%
  \vfill
  \hfill
  
\begin{minipage}{0.54\textwidth}
\begin{flushleft} \large
% texte
\includegraphics[width=6cm]{telecomnancy.jpg}


\end{flushleft}
\end{minipage}
\begin{minipage}{0.4\textwidth}
\begin{flushright} \large
\includegraphics[width=6cm]{univ-lorraine.jpg}
\end{flushright}
\end{minipage}
  \cleardoublepage
  }
  
\makeatother

\title{English Project \\ \textbf{TOEIC Trainer}}
\author{\textsc{Clément Schanen} \\ \textsc{Thibaut Smith}}
\date{2013-2014}

    %\renewcommand*\thesection{\arabic{section}}
    
    
\makeatletter
\g@addto@macro\@floatboxreset\centering
\makeatother


\begin{document}
\pagestyle{empty} % removes page numbering

\maketitle % implicit pagebreak

\pagestyle{plain}
\setcounter{page}{2}

\tableofcontents
\section*{Introduction}
As part of TELECOM Nancy's English course, we, \textsc{Clément Schanen} and
\textsc{Thibaut Smith} tried our best to come up with an interesting and
useful product for the students of TELECOM Nancy.

One of our objectives for this project was to make a sustainable product that
would allow for the community using it to refine it themselves and adding
materials to study.

\pagebreak
\section{Website's Handbook}
% 2-5 pages
% Présentation de l'outil, mode d'emploi

The product that we came up with is a website\cite{TOEICTrainer}. In order to
complete the objectives we wanted, we thought of a website that could reach
everyone at school, and beyond. The sources of the project are free and
opensource for anyone to review/edit/fork\footnote{Forking is when someone
decides to take a project's code to create he's own project with it.}, on
GitHub\cite{github_tc}.

\begin{figure}[here]
\includegraphics[scale=0.35]{homepage.png}
\caption{The TOEICTrainer home page\cite{TOEICTrainer}, available at \url{http://toeic.fslhome.org/app.php/toeic/}.}
\label{homepage}
\end{figure}

You can see on the Figure \ref{homepage} the website. It is currently available on
internet and should be working. The homepage describes to any user seeing the site what is the purposes of the site: \textit{providing a training platform for the ones who needs it}.

Please note the presence of the \textbf{navigation bar} on the top, which is the preferred way to move around. With it, you can access the exercises and the submitting pages for new content.


\subsection{What you can do with this website}

The main goal of this website is helping people improve their
English level by doing TOEIC-like exercises. Our exercises are
divided in two parts: the \textbf{listening part} and the
\textbf{reading part}, as in the real TOEIC. And every part
is also divided in two exercises, so a user can do four different
exercises: \textbf{Photographs}, \textbf{Questions/Answers on Audio} and
on \textbf{Documents}, and \textbf{Incomplete documents}.

Moreover, the website is collaborative: anyone can submit new content (text, audio or picture) to build new exercises (Questions, Sound).

\subsection{Listening exercises}

The listening part is constituted of two different exercises inspired by the real TOEIC test :

\begin{itemize}
\item The first exercise consists in looking at a photograph and hearing three audio description. The goal of this exercise is to find the audio description that best corresponds to the photograph.

\begin{figure}[here]
\includegraphics[scale=0.45]{captureToeic.png}
\caption{Select the description that best correspond to the photography}
\end{figure}

\item The second exercise is constituted of a question and three possible answers to this question. The user has to hear the question and the three possible answers and find the best answer regarding the question.
\end{itemize}

\subsection{Reading exercises}
The TOEIC has three kinds of exercises in the reading session. We implemented
two of them:

\begin{itemize}
\item Fill the blanks in an incomplete sentences
\item \textbf{Fill the blanks in a document}
\item \textbf{Use documents to answer questions about them}
\end{itemize}

The bold lines are the exercises you can find on the website.

\begin{figure}[here]
\includegraphics[scale=0.45]{rex1end.png}
\caption{Fill the blanks in a document}
\label{reading.ex1.end}
\end{figure}



\begin{figure}[here]
\includegraphics[scale=0.4]{rex2end.png}
\caption{Fill the blanks in a document}
\label{reading.ex2.end}
\end{figure}

You can see on Figure \ref{reading.ex2.end} the display of the correction. You
can add as many questions as you wish for a written document.


\subsection{Create your own exercises}

This website allow you to participate by writing documents or uploading new
sounds and photographs for the exercises. It is also possible to record your
voice (in english) in order to create new sounds for the exercises. Moreover,
you can also participate by creating new exercises using already existing
contents.

The pages to add contents are visible on the navigation bar
under the "\textit{Participate!}" button.

Because people will usually not bother submitting contributions to the site, we made it so \textbf{adding a little contribution can be used on as much exercises as possible}. In the end, contributors will have less forms to view, but maybe more data to input at once: for example, when submitting a sound recording, the audio transcript needs to be entered. It can then be used in written questions.

\pagebreak

\section{Documentation technique}
% 2-5 pages

Ce projet fut compliqué à mettre en oeuvre, pour plusieurs raisons. D'abord, le temps nous manquait, en partie à cause des plusieurs autres projets et examens en parallèle. Et surtout car nous avons souhaité utiliser des technologies \textbf{nouvelles}, \textbf{puissantes}, et \textbf{largement reconnues} dans le monde du travail qui nous étaient inconnues jusqu'à alors. Nous voulions développer nos connaissances et notre expérience. 

Nous avons aussi pu développer nos connaissances dans des langages de
programmation tels que PHP\cite{php}, Javascript\cite{javascript} et
CSS\cite{css}. Pour apprendre le Javascript, on est allé sur Codecademy\cite{codecademy}.




\subsection{Mise en oeuvre et développement}

Nous pouvons découper l'architecture du site en deux parties bien précises :
\begin{itemize}
	\item La \textbf{partie dynamique}, ou \textbf{back-end}, qui s'occupe de prendre en compte les actions
	 de l'utilisateur via les exercices qu'il peut crééer et utiliser pour
	 s'entra\^iner.
	 
	 \item La \textbf{partie statique}, ou \textbf{front-end}, qui contrôle l'affichage et le rendu du design.
\end{itemize}

Nous avons choisi l'outil \textbf{Symfony}\cite{Symfony} qui est un
\textbf{framework}\footnote{Un ensemble de codes sources pr\^et à \^etre utiliser.} qui implémente énormément de fonctionnalités pratiques
lorsque l'on développe un site web. Par exemple, Symfony gère automatiquement
l'affichage des formulaires, quelque chose qui est particulièrement g\^enant et
qui prend du temps. Voici une liste non exhaustives de ce qu'il peut faire :

\begin{itemize}
	\item Gestion des utilisateurs simplifiées.
	\item Gestion d'une architecture \textbf{Modèle-Vue-Controleur} simple à
	mettre en place.
	\item Gestion de la base de données simple à mettre en place.
\end{itemize}

Mais le principale avantage de Symfony est avant tout la documentation et la
durabilité du code : un code fonctionnant sur Symfony peut être écrit de
manière standard \textbf{que tout développeur Symfony} peut comprendre. Cela
signifie que ce code peut être fourni à quiconque connaissant Symfony, et
comprendra parfaitement l'architecture du code et la manière dont sont
articulés les différents élements du projet.

Pour \textbf{TOEIC Trainer}, les principaux élements sont :

\begin{itemize}
	\item Le fait de pouvoir s'entraîner sur des exercices.
	\item Le fait de pouvoir créer des exercices.
\end{itemize}

Pour pouvoir créer un design plus rapidement, nous avons intégré le design Bootstrap\cite{Bootstrap} avec l'outil BraincraftedBootstrap\cite{BraincraftedBootstrapBundle} qui permet de relier Bootstrap et Symfony plus facilement.

\subsubsection{Gestion des exercices}

La gestion des exercices prend en compte la création de nouveaux exercices
ainsi que l'affichage et la correction des exercices lorsque l'on s'entraîne.

L'aspect affichage et la correction des exercices est complètement différent de
la création d'un exercice, ce qui nous a permis de bien séparer ces taches.

Pour cette partie, nous avons rajouté l'utilisation du Javascript qui permet
d'avoir des évènements qui se passent \textbf{après} le chargement de la page.
C'est en particulier ce qui permet un affichage agréable des exercices et de leur correction, et lorsqu'on créé de nouveaux exercices.

La lecture de la musique se fait via HTML5, et nous avons appris à gérer
l'enregistrement et l'utilisation du microphone de l'ordinateur avec l'outil
\textbf{Recorder.js}\cite{recorderjs}.

\subsection{Problèmes rencontrés}
Nous avons eu énormement de soucis liés aux technologies que nous ne
connaissions pas. Par exemple, l'apprentissage de Symfony a nécessité
la(re-)lecture d'un livre, de la documentation du site officiel, et d'un site
d'échange d'informations\cite{so} sur le sujet de la programmation lorsque l'on
avait besoin d'aide.

Symfony dispose d'un channel dédié\cite{freenode} pour pouvoir discuter à des
connaisseurs de Symfony, chose que l'on a pu essayer sans grands résultats.

Voici une liste des problèmes de développement qui nous est arrivé :

\begin{itemize}
	\item \textbf{Configuration des formulaires}. L'automatisation des 
	formulaires de Symfony a un prix, et c'est sa complexité. Heureusement,
	la documentation\cite{docform} est très complète.
	\item Personnalisation des formulaires. 
	\item \textbf{Utilisation de la base de données et des relations entre
	entités stockés}.
\end{itemize}

Ceux en gras sont ceux que nous avons su résoudre de manière suffisamment complète.

\section*{Conlusion}
This project made us learn a great deal of things in web development, and we
can be proud to say that we have now a better insight of the most famous tools
and programming languages currently used in business.
 
We tried analysing what were the requirements of such a project to come up with
an interesting result that would not only help people studying for the TOEIC
examination, but also a product that would not expire as fast as a project gone
still. This is how we thought of giving everyone the possibility to submit new
content and giving the ones that are more at ease with the english language.


\bibliographystyle{plain} % biblio style
\bibliography{biblio} % to load biblio.bib


\end{document}
